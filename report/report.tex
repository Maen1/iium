\documentclass[12pt]{report}
    \usepackage{url}
    \usepackage{xcolor}
    \usepackage{caption}
    \usepackage{pgfgantt}
    \usepackage{mathptmx} 
    \usepackage{graphicx}
    \usepackage{subcaption}
    \usepackage[nottoc,notlof,notlot]{tocbibind} 
    \renewcommand\bibname{References}            
    \usepackage[a4paper, total={6in, 8in}]{geometry}
   
    \definecolor{tail}{RGB}{26, 188, 156}
    \geometry{margin=1in}
    \parindent0pt  
    \parskip10pt             
    \raggedright   
    \title{Sign language recognition using deep learning}
    \author{ Name: MHD Khaled Maen\\
        Matric No: 1523592 \\ 
        [1.5cm]
        Supervised by\\
        Assoc. Prof. Dr. Amelia Ritahani \\}  
    \begin{document}
    @Article{Frajtag12017,
        author  = {Jasenka Broz Frajtag1 and Jagoda Doko Jelinic2},
        title   = {Communication Problems and Quality of Life People with Hearing Loss},
        journal = {Glob J Otolaryngol},
        year    = {2017},
    }
    \maketitle
    \pagenumbering{roman}                   
    \setcounter{page}{2}                    
    \tableofcontents
    \newpage
    \pagenumbering{arabic}
    \chapter{Introduction} 
        \section{Background}
            \paragraph{}
                Communication is a process of sending and receiving information among people. 
                Humans communicate with others by a lot of ways but the most effective way is 
                face-to-face communication. Many people believe that the significance of communication 
                is like the importance of breathing. Indeed, communication facilitates the spread of knowledge
                and forms relationships between people.
            
                Deep learning added a huge boost to the already rapidly developing field of computer vision.
                With deep learning, a lot of new applications of computer vision techniques have been introduced
                and they are now becoming parts of our everyday live.
            
                Alongside  with  the power of today's computers, there are now various algorithms that were developed 
                to enable the computers to perform tasks such as object tracking and pattern recognition. 
                
                In this study, the focus will be on hand gestures detection and live tracking.

        \section{Problem Statement}
            \paragraph{}
                Communication difficulties arising from damage to hearing
                directly affect quality of life. Difficulties in communication may
                result in deviations in the emotional and social development and
                can have a significant impact on the quality of life of every person.
                It is well recognized that hearing is critical to speech and language
                development, communication, and learning. People with listening
                difficulties due to hearing loss or auditory processing problems
                continue to be an underidentified and underserved population. The
                earlier the problem is identified and intervention begun, the less
                serious the ultimate impact \cite{Frajtag12017}.
                The communication between hearing-impaired and other people is
                a huge gab need to be filled up. In order to overcome this challenge 
                many researches and products have been developed to solve these problem, 
                but there is a lot to be enhanced.
        
        \section{Objectives}
            \begin{itemize}
                \item To study sign language gestures.
                \item To develop a new hand gesture into voice algorithm.
                \item To construct a hand gesture into voice model.
            \end{itemize}
        
        \section{Scope}
            \paragraph{}
                This research aims to develop a sign language recognition algorithm,
                and converting it into voice.
        \section{Significance}
            \paragraph{}
                give the hearing-impaired the power to communicate with hearing ones, 
                in order to make a strong connected community.

        \section{Timeline}
            \begin{center}
                \begin{ganttchart}[
                    expand chart=\textwidth,
                    bar/.append style={draw=none, fill=tail},
                    hgrid style/.style={draw=black!5, line width=.75pt},
                    vgrid={*1{draw=black!5, line width=.75pt}},
                    ]{1}{14}
                    \gantttitle{Week}{14} \\
                    \gantttitlelist{1,...,14}{1} \\
                    \ganttbar{title selectin}{1}{3}  \\
                    \ganttbar{Overall System Review}{2}{7}  \\
                    \ganttbar{Literature Review}{4}{12}  \\
                    \ganttbar{System Design}{8}{11}  \\
                    \ganttbar{Design \& Prototype}{9}{12}  \\
                    \ganttbar{Simulation}{7}{13}  \\
                    \ganttbar{Report Writing}{5}{13}  \\
                    \ganttbar{Submission}{13}{13}  \\
                \end{ganttchart}
            \end{center}
    \chapter{Literature review}
        \section{Introduction}
            \paragraph{}
                This chapter includes reviews of other previous researcher
                and their proposed methods they used in implementing deep learning
                to recognize hand gestures. These researches will help to grasp the knowledge
                to achieve the project's objectives. 
                
        \section{Previous works}
            \paragraph{}
                Peijun Bao, Ana I. Maqueda, Carlos R. del-Blanco, and Narciso Garcia, 
                (2017) proposed a Deep convolutional neural network algorithm for hand-gesture 
                recognition without hand localisation, since the hands only occupy about 10\% of 
                the image. They used a combination of 9 convolution layers, 3 fully connected layers, 
                interlaced with ReLU(Rectified Linear Unit) and dropout layers as shown in 
                figure \ref{fig:tiny_architecture}. Alongside this architecture the apply some image 
                processing techniques to have sufficient computation efficiency and memory requirement.
                According to the paper the accuracy achieved was 97.1\% in the images with simple backgrounds
                and 85.3\% in the images with complex backgrounds.However, the main disadvantage of of 
                the proposed algorithm is the training set which only includes 7 different gestures,
                and it tends to have bad accuracy with complex backgrounds.
             
                    \begin{figure}
                        \centering
                        \begin{subfigure}[a]{0.5\textwidth}
                          \includegraphics[width=\textwidth]{./images/tiny_a.png}
                        \end{subfigure}
                        \begin{subfigure}[b]{0.3\textwidth}
                            \includegraphics[width=\textwidth]{./images/tiny_b.png}
                        \end{subfigure}
                        \caption{Architecture of the proposed deep CNN }\label{fig:tiny_architecture}
                    \end{figure}

                \newpage

            \paragraph{}
                G.Anantha Rao, K.Syamala , P.V.V.Kishore, A.S.C.S.Sastry, (2018)
                proposed a CNN architecture for classifying selfie sign language gestures. 
                The CNN architecture is designed with four convolutional layers. Each convolutional 
                layer with different filtering window sizes as shown in figure \ref{fig:selfie}  
                They had a dataset with five different subjects performing 200 signs in 5 different viewing angles 
                under various background environments. Each sign occupied for 60 frames or images in a video.
                The proposed model performed training on 3 batches to test the robustness of different training mode 
                using caffe deep learning framework. However, the result accuracy was 92.88\% need more training and improvements. 
    
                    \begin{figure}[h]
                        \centering
                        \includegraphics[width=\textwidth]{./images/selfie.png}
                        \caption{Proposed Deep CNN architecture}
                        \label{fig:selfie}
                    \end{figure}

                 \newpage
            
            \paragraph{}
                Soeb Hussain, Rupal Saxena Xie Han, Jameel Ahmed Khan, Prof. Hyunchul Shin, (2018)
                introduced a CNN based classifier  trained through the process of transfer learning
                over a pretrained convolutional neural network which is trained on a large dataset.
                We are using VGG16 figure \ref{fig:vgg16} as the pretrained model.
                The According to the paper the accuracy was 93.09\%,while using AlexNet 
                figure \ref{fig:alexnet} was 76.96\%. the same problem here with the other papers 
                which is the small number of sign that begin trained on 7 signs, and the accuracy
                need to be improved as well.

                    \begin{figure}[h]
                        \centering
                        \includegraphics[width=\textwidth]{images/vgg16.png}
                        \caption{VGG16 architecture. Retrieved from www.cs.toronto.edu}
                        \label{fig:vgg16}
                    \end{figure}
                    \begin{figure}[h]
                        \centering
                        \includegraphics[width=\textwidth]{images/alexnet.png}
                        \caption{VGG16 architecture. Retrieved from www.saagie.com}
                        \label{fig:alexnet}
                    \end{figure}

                \newpage

            \section{Summary}
                \paragraph{}
                    This chapter illustrates some works have been done previously on
                    hand gesture and sign language recognition using deep learning.
                    Table \ref{table:summary} the Summary of the literature review.
                    
                    \begin{center}
                        \begin{table}[h]
                            \caption{Summary of the literature review}
                            \begin{tabular}{ |p{7cm}|p{2cm}|p{2cm}|p{3cm}| }
                                \hline
                                Title & Year & Accuracy & Software\\
                                \hline
                                Tiny Hand Gesture Recognition without Localization via a Deep Convolutional Network & 2017 & 97.1\%& CNN \\
                                \hline
                                Deep Convolutional Neural Networks for Sign Language Recognition & 2018 & 92.88\% & CNN \\
                                \hline
                                Hand Gesture Recognition Using Deep Learning & 2017 & 93.09\% & CNN VGG16 \\
                                \hline
                            \end{tabular}
                            \label{table:summary}
                        \end{table}
                    \end{center}
        \newpage
        
                    
        \bibliographystyle{abbrv}
        \bibliography{scope.bib}
    \end{document}
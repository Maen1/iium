\documentclass[12pt]{report}
    \usepackage{url}
    \usepackage{caption}
    \usepackage{mathptmx} 
    \usepackage{graphicx}
    \usepackage{subcaption}
    \usepackage[a4paper, total={6in, 8in}]{geometry}
    \geometry{margin=1in}
    \parindent0pt  
    \parskip10pt             
    \raggedright   
    \title{Sign language recognition using deep learning}
    \author{ Name: MHD Khaled Maen\\
        Matric No: 1523592 \\ 
        [1.5cm]
        Supervised by\\
        Assoc. Prof. Dr. Amelia Ritahani \\}  
    \begin{document}
    \maketitle
    \pagenumbering{roman}                   
    \setcounter{page}{2}                    
    \tableofcontents
    \newpage
    \pagenumbering{arabic}
    \chapter{Introduction} 
        \section{Background}
        \paragraph{}
        Communication is a process of sending and receiving information among people. 
        Humans communicate with others by a lot of ways but the most effective way is  face-to-face communication, 
        Many people believe that the significance of communication is like the importance of breathing. 
        Indeed, communication facilitates the spread of knowledge and forms relationships between people.
    
        Deep learning added a huge boost to the already rapidly developing field of computer vision.
        With deep learning, a lot of new applications of computer vision techniques have been introduced and are now becoming parts of our everyday lives.
    
        alongside  with  the power of today's computers, there are now various algorithms that were developed to enable a
        computers to perform tasks such as object tracking and pattern recognition. 
        
        In this study, the focus will be on hand gestures detection and live tracking. 
        \section{Problem Statement}
        \paragraph{}
        The communication between disabled and other people is
        a really huge gab need to be filled up.
        In order to overcome this challenge many researches and products have been developed to solve these problem, 
        but there is a lot to be enhanced.
        
        \section{Objectives}
        \begin{itemize}
            \item To study sign language gestures.
            \item To develop a new hand gesture into voice algorithm.
            \item To construct a hand gesture into voice model.
        \end{itemize}    
    \chapter{Literature review}
        \section{Introduction}
            \paragraph{}
            This chapter includes reviews of other previous researcher
            and their proposed methods they used in implementing deep learning
            to recognize hand gestures. These researches will help to grasp the knowledge
            to achieve the project's objectives. 
        \section{Previous works}
            \paragraph{}
            Peijun Bao, Ana I. Maqueda, Carlos R. del-Blanco, and Narciso Garcia, 
            (2017) proposed a Deep convolutional neural network algorithm for hand-gesture recognition
            without hand localisation, since the hands only occupy about 10\% of the image.
            They used a combination of 9 convolution layers, 3 fully connected layers, interlaced with ReLU 
            (Rectified Linear Unit) and dropout layers as shown in figure ~\ref{fig:tiny_architecture}.
            alongside this architecture the apply some image processing techniques
            to have sufficient computation efficiency and memory requirement.
            According to the paper the accuracy achieved was 97.1\% in the images with simple backgrounds
            and 85.3\% in the images with complex backgrounds.
            However, the main disadvantage of of the proposed algorithm
            is the training set which only includes 7 different gestures,
            and it tends to have bad accuracy with complex backgrounds.
             
            
                \begin{figure}
                    \centering
                    \begin{subfigure}[a]{0.5\textwidth}
                    \includegraphics[width=\textwidth]{./images/tiny_a.png}
                    \end{subfigure}
                    \begin{subfigure}[b]{0.3\textwidth}
                    \includegraphics[width=\textwidth]{./images/tiny_b.png}
                    \end{subfigure}
                    \caption{Architecture of the proposed deep CNN }\label{fig:tiny_architecture}
                \end{figure}
                \newpage
                \paragraph{}
                G.Anantha Rao, K.Syamala , P.V.V.Kishore, A.S.C.S.Sastry, (2018)
                proposed a CNN architecture for classifying selfie sign language gestures. 
                The CNN architecture is designed with four convolutional layers. Each convolutional 
                layer with different filtering window sizes as shown in figure \ref{fig:selfie}  
                They had a dataset with five different subjects performing 200 signs in 5 different viewing angles 
                under various background environments. Each sign occupied for 60 frames or images in a video.
                The proposed model performed training on 3 batches to test the robustness of different training mode 
                using caffe deep learning framework. However, the result accuracy was 92.88\% need more training and improvements. 
    
                    \begin{figure}
                        \centering
                        \includegraphics[width=\textwidth]{./images/selfie.png}
                        \caption{Proposed Deep CNN architecture}\label{fig:selfie}
                    \end{figure}
    \end{document}
\documentclass[a4paper,man,natbib]{apa6}
\usepackage[english]{babel}
\usepackage[utf8x]{inputenc}
\usepackage{amsmath}
\usepackage{graphicx}
\usepackage[colorinlistoftodos]{todonotes}
\usepackage[style=apa,sortcites=true,sorting=nyt,backend=biber]{biblatex}
\DeclareLanguageMapping{american}{american-apa}
\addbibresource{bibliography.bib}

\title{SIGN LANGUAGE RECOGNITION USING DEEP LEARNING}
\shorttitle{s}
\author{You}
\affiliation{df}

\abstract{
    Communication is an essential part of our life. Unfortunately, some of us were born in
various types of disability such as deaf, since hearing impaired people can’t listen they can’t
learn to speak so they developed a new communication why to interact with other people by
using distinct hand gestures, which wasn’t enough to overcome this issue, even now with all
technologies and tool still challenging problem to solve. For the mentioned reason, the
intention of the proposed research is to improve an ordinary model to translate the hand
gestures of the sign language into voice.
In that, Deep learning is remarkably serviceable for this mission, firstly by identifying the
hand in the video frame by using Convolutional neural network algorithm then states the
sound that matches the sign.
}

\begin{document}
\maketitle

\section{Test}
Communication is an essential part of our life. Unfortunately, some of us were born in
various types of disability such as deaf, since hearing impaired people can’t listen they can’t
learn to speak so they developed a new communication why to interact with other people by
using distinct hand gestures, which wasn’t enough to overcome this issue, even now with all
technologies and tool still challenging problem to solve. For the mentioned reason, the
intention of the proposed research is to improve an ordinary model to translate the hand
gestures of the sign language into voice.
In that, Deep learning is remarkably serviceable for this mission, firstly by identifying the
hand in the video frame by using Convolutional neural network algorithm then states the
sound that matches the sign.
\end{document}